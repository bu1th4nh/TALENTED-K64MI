\documentclass[16pt, hyperref={unicode}]{beamer}
\usepackage[utf8]{vietnam} % Sử dụng tiếng việt
\usepackage{graphicx} % Cho phép chèn hỉnh ảnh
\usepackage{amsmath}
\usepackage{amsfonts}
\usepackage{amssymb}
\usepackage{graphics} % thư viện để hiển thị ảnh
\usepackage{caption} % thư viện để đặt caption
\usepackage{booktabs} % thư viện để vẽ table
\usepackage{hyperref} % thư viện để link tới các chương
\usepackage[boxruled, lined]{algorithm2e}
\usepackage{xcolor}
 

\usetheme{CambridgeUS}
\usecolortheme{crane}
%=======================================================
% Hình vẽ
\graphicspath{{./rpt-img/}}


% Phụ lục 
% \usepackage[toc,page]{appendix} % Thêm phụ lục
% \renewcommand{\appendixname}{Phụ lục}
% \renewcommand{\appendixtocname}{Phụ lục}
% \renewcommand{\appendixpagename}{Phụ lục}

% Tài liệu tham khảo
\usepackage[backend=biber]{biblatex}
\addbibresource{bibliography/report-bib.bib}
\DefineBibliographyStrings{english}{%
bibliography = {Tài liệu tham khảo},
references = {Tài liệu tham khảo},
}



% Dãn cách dòng & đánh số phần
% \setlength{\parskip}{6pt}
% \setlength{\parindent}{15pt}
% \renewcommand{\baselinestretch}{1.5}
\renewcommand{\thesection}{\arabic{section}}


% Tên và tác giả
\title{Các phương pháp tìm gần đúng ma trận nghịch đảo}
\author{\textbf{Bùi Tiến Thành - MSSV 20190081}  \\\textbf{Chu Thị Ngân - MSSV 20195904} \\}
\logo{\includegraphics[height=1.5cm]{LOGO_HUST.png}}
\institute{CTTN Toán tin K64}
\date{Tháng Mười một, 2020}


% Đầu trang và chân trang
% \pagestyle{fancy}
% \fancyhf{}
% \lhead{\rightmark}
% \rfoot{CTTN Toán tin K64 }
% \lfoot{Bùi Tiến Thành \texttt{(thanh.bt190081@sis.hust.edu.vn)} \\Chu Thị Ngân \texttt{(ngan.ct195904@sis.hust.edu.vn)}}
% \rhead{Trang \thepage}
% \renewcommand{\headrulewidth}{2pt}
% \renewcommand{\footrulewidth}{2pt}
% đánh số cho caption và footline
\setbeamertemplate{caption}[numbered]
\setbeamertemplate{footline}[frame number]


\AtBeginSection[]
{
  \begin{frame}
    \frametitle{Mục lục}
    \tableofcontents[currentsection]
  \end{frame}
}


%=======================================================
\begin{document}

    \begin{frame}
        \maketitle 
    \end{frame}
    
    \section{Đặt vấn đề}
    \subsection{Tại sao phải giải gần đúng?}
    \begin{frame}
        \frametitle{Tại sao phải giải gần đúng?}

        \par Các phương pháp giải đúng:
        \begin{itemize}
            \item Bắt buộc phải duyệt toàn bộ các phần tử của ma trận và phải lưu toàn bộ ma trận trong bộ nhớ, đặc biệt với các thuật toán đệ quy
            \item \textcolor{red}{Không thể kiểm soát sai số tính toán} do giới hạn tính toán của máy tính
        \end{itemize}

        \par $ \Rightarrow $ \textbf{Giải pháp?}
    \end{frame}

    \subsection{Các phương pháp đưa ra}
    \begin{frame}
        \frametitle{Các phương pháp đưa ra}
        \begin{itemize}
            \item Phương pháp Newton
            \item Phương pháp lặp Jacobi
            \item Phương pháp lặp Gauss-Seidel
        \end{itemize}
    \end{frame}

    
    \section{Phương pháp Newton}
\subsection{Ý tưởng và công thức lặp}
    \begin{frame}{Ý tưởng và công thức lặp}

        \begin{block}{Ý tưởng}
            Từ phương pháp Newton cho $ax = 1$:
            $$ x_{n+1} = x_{n} * (2 - ax_{n}) \text{ với } n \in \mathbb{N} $$
        \end{block}

        Áp dụng vào ma trận với E là ma trận đơn vị cùng cấp:
        \begin{block}{Công thức lặp}
            $$ X_{k+1} = X_{k}(2E - AX_{k}) \text{ với } k \in \mathbb{N} $$
        \end{block}

    \end{frame}

\subsection{Điều kiện hội tụ và công thức sai số}

    \begin{frame}
        \frametitle{Điều kiện hội tụ và công thức sai số}
        
        \begin{block}{Điều kiện hội tụ}
            $$ \left\lVert A^{-1} - X_{k} \right\rVert \leq \left\lVert A^{-1} \right\rVert \left\lVert E - AX_{0} \right\rVert ^{2^{k}} \xrightarrow{k \to \infty} 0 \text{ nếu }  \left\lVert E - AX_{0} \right\rVert < 1 $$
        \end{block}

        \begin{block}{Công thức sai số}
            Khi $\left\lVert E - AX_{0} \right\rVert \leq q < 1$: 

            $$ \left\lVert A^{-1} - X_{k} \right\rVert \leq \frac{\left\lVert X_{0} \right\rVert}{1 - q} q^{2^{k}}  $$
        \end{block}
    
    \end{frame}

    \subsection{Thuật toán và chương trình}
    \begin{frame}
        \frametitle{Input, output}
        \begin{itemize}
            \item Input: Ma trận $A$, xấp xỉ đầu $X_{0}$ và sai số $\varepsilon$. Giả thiết coi như các điều kiện hội tụ thỏa mãn.
            \item Output: Ma trận xấp xỉ $A^{-1}$
            \item Code:
            \begin{center}
                \includegraphics[scale = 0.4]{newton-code-qr.png}

                \url{https://github.com/bu1th4nh/TALENTED-K64MI/blob/master/MI3040/report-code/newton.py}
            \end{center}
        \end{itemize}
    \end{frame}

    \begin{frame}[fragile]  
        \frametitle{Mã giả}

        \IncMargin{1em}\begin{algorithm}[H]
            \caption{Phương pháp Newton tìm ma trận nghịch đảo \label{IR}}
            \KwIn{Ma trận $A, X_{0}, \varepsilon$}
            \KwOut{Ma trận $X^{*} = A^{-1}$ là ma trận nghịch đảo}
            \SetAlgoLined            
            \Begin{
                Nhập $A, X_{0}, \varepsilon$\;
                \BlankLine
                $ q \longleftarrow \left\lVert E - AX_{0} \right\rVert $\;
                $ q2k \longleftarrow q $\;
                $ X \longleftarrow X_{0}   $\;
                \BlankLine
                \While{$ \frac{\left\lVert X_{0} \right\rVert * q2k }{1 - q} > \varepsilon $}{    
                    $ X \longleftarrow X(2E - AX) $\;
                    $ q2k \longleftarrow q2k^{2} $\;
                }
                \BlankLine
                Đưa ra $X$ chính là ma trận nghịch đảo\;
            }
        \end{algorithm}\DecMargin{1em}
    \end{frame}


    \section{Giới thiệu các phương pháp tìm nghịch đảo ma trận chéo trội}
    \subsection{Ý tưởng chung}
    \begin{frame}
        \frametitle{Ý tưởng chung}

        \begin{block}{Ý tưởng}
            Đưa phương trình $AX = E$ về dạng công thức lặp sau:
            $$ X_{k+1} = BX_{k} + D $$ 
        \end{block}

        Ta cũng chứng minh được nếu $ \left\lVert B \right\rVert < 1 $ thì với giá trị xấp xỉ đầu bất kì thì $X$ luôn hội tụ về nghiệm (Phương pháp lặp đơn).

        $\longrightarrow$ Vậy làm thế nào để $ \left\lVert B \right\rVert < 1 $?
    \end{frame}

    \subsection{Các phương pháp} 
    \begin{frame}
        \frametitle{Các phương pháp được lựa chọn}
        
        \begin{itemize}
            \item Phương pháp lặp Jacobi (dựa trên phương pháp lặp đơn)
            \item Phương pháp lặp Gauss - Seidel (dựa trên phương pháp lặp Siedel)
        \end{itemize} 
    \end{frame}

    \input{slides/slide-sect4.tex}
    \section{Phương pháp lặp Gauss-Seidel}
\subsection{Công thức lặp và điều kiện hội tụ }
    \begin{frame}
        \frametitle{Công thức lặp và điều kiện hội tụ}

        Đặt $B = E - TA$ với $T = diag\{\frac{1}{A_{11}}, \frac{1}{A_{22}},..., \frac{1}{A_{nn}}\} $ tương tự phương pháp Jacobi
        \begin{block}{Công thức lặp - Phương pháp Gauss-Seidel nguyên bản \cite{giaotrinhgiaitichso}}
            $$ X_{i}^{(k+1)} = \sum\limits_{j=1}^{i-1} B_{ij}X_{j}^{(k+1)} + \sum\limits_{j=i+1}^{n} B_{ij}X_{j}^{(k)} + T_{i}$$
            với mọi $i = \overline{1,n}$ và $A_{i}$ là dòng $i$ của ma trận $A$
        \end{block}

        \begin{block}{Mở rộng: Hệ số điều chỉnh cho phương pháp Gauss-Seidel \cite{nummethod4}}
            $$ X_{i}^{k+1} = (1 - \omega)X_{i}^{k} + \omega\left[ \sum\limits_{j=1}^{i-1} B_{ij}X_{j}^{(k+1)} + \sum\limits_{j=i+1}^{n} B_{ij}X_{j}^{(k)} + T_{i}\right] $$

            với $\omega$ là hệ số điều chỉnh (relaxation factor)
        \end{block}

        \par \textbf{Điều kiện hội tụ:} Tương tự phương pháp \textbf{Jacobi}, tức là $A$ phải chéo trội.
        
    \end{frame}

\subsection{Công thức sai số}
    \begin{frame}
        \frametitle{Trường hợp chéo trội hàng}
        Cho: 
        $$ q = \max\limits_{1 \leq i \leq n} \frac{\sum\limits_{j=i}^{n} \left\lvert B_{ij} \right\rvert}{1 - \sum\limits_{j=1}^{i-1} \left\lvert B_{ij} \right\rvert} \leq \left\lVert B \right\rVert_{\infty} < 1 $$
        
        \begin{block}{Công thức sai số \cite{giaotrinhgiaitichso}}
            $$ \left\lVert X_{k} - X^{*} \right\rVert_{\infty} \leq \frac{q}{1 - q} \left\lVert X_{k} - X_{k - 1} \right\rVert_{\infty}  $$
            $$ \left\lVert X_{k} - X^{*} \right\rVert_{\infty} \leq \frac{q^{k}}{1 - q} \left\lVert X_{1} - X_{0} \right\rVert_{\infty}  $$
        \end{block}
    \end{frame}

    \begin{frame}
        \frametitle{Trường hợp chéo trội cột}
        Cho: 
        $$ q = \max\limits_{1 \leq i \leq n} \frac{\sum\limits_{j=1}^{i} \left\lvert B_{ji} \right\rvert}{1 - \sum\limits_{j=i+1}^{n} \left\lvert B_{ji} \right\rvert} \leq \left\lVert B \right\rVert_{1} < 1 $$ 
        $$ S = \max\limits_{1 \leq i \leq n} \sum\limits_{j=i+1}^{n} \left\lvert B_{ji} \right\rvert $$
        
        \begin{block}{Công thức sai số\cite{giaotrinhgiaitichso}}
            $$ \left\lVert X_{k} - X^{*} \right\rVert_{1} \leq \frac{q}{(1 - S)(1 - q)} \left\lVert X_{k} - X_{k - 1} \right\rVert_{1}  $$
            $$ \left\lVert X_{k} - X^{*} \right\rVert_{1} \leq \frac{q^{k}}{(1 - S)(1 - q)} \left\lVert X_{1} - X_{0} \right\rVert_{1}  $$
        \end{block}
    \end{frame}

\subsection{Thuật toán và chương trình}
    \begin{frame}[fragile, label = gauss_seidel.algo]  
        \frametitle{Thuật toán chính}
        \IncMargin{1em}\begin{algorithm}[H]
            \caption{Phương pháp Gauss-Seidel tìm ma trận nghịch đảo \label{IR}}
            \KwIn{Ma trận $A$ chéo trội, sai số $\varepsilon$, hệ số điều chỉnh $\omega$}
            \KwOut{Ma trận $X^{*} = A^{-1}$ là ma trận nghịch đảo}
            \SetAlgoLined   
            \Begin{
                Nhập $A, \varepsilon$\;
                \BlankLine
                $ p \longleftarrow $ \hyperlink{jacobi.checkDom}{checkDomination}$(A) $\;
                $ T \longleftarrow diag\{ \frac{1}{A_{11}}, \frac{1}{A_{22}}, ..., \frac{1}{A_{nn}} \} $\;
                $ B \longleftarrow E - TA $\;
                $ S \longleftarrow $ \hyperlink{gauss_seidel.get_S_coeff}{getSCoeff}$(B, p) $\;
                $ q \longleftarrow $ \hyperlink{gauss_seidel.get_q_coeff}{getqCoeff}$(B, p) $\;
                \BlankLine
                $ X^{*} \longleftarrow $ \hyperlink{gauss_seidel.iterate}{iterate}$(X_0 \leftarrow A, B, T, S, q, p, \omega, \varepsilon) $\;
                Trả về $X^{*}$ là ma trận nghịch đảo\;
            }         
        \end{algorithm}\DecMargin{1em}
        \hyperlink{gauss_seidel.code}{\textcolor{blue}{Chuyển sang phần code}}
    \end{frame}

    
    \begin{frame}[fragile, label = gauss_seidel.get_S_coeff]
        \IncMargin{1em}\begin{function}[H]
            \caption{getSCoeff(A, p)}
            \KwIn{Ma trận $A$, giá trị kiểm tra $p$}
            \KwOut{ $ S = 0 $ nếu $p = 1$, $\max\limits_{1 \leq i \leq n} \sum\limits_{j=i+1}^{n} \left\lvert A_{ji} \right\rvert $ nếu $p = -1$}
            \Begin{
                \lIf{$ p = 1 $}{\KwRet{0}}
                \BlankLine
                $S \longleftarrow 0$\;
                \lFor{$i=1$ \KwTo n}{$ S \longleftarrow max(S, \sum\limits_{j=i+1}^{n} \left\lvert A_{ji} \right\rvert) $}
                \KwRet{$S$}
            }
        \end{function}\DecMargin{1em}

        \hyperlink{gauss_seidel.algo}{\textcolor{blue}{Trở về thuật toán chính}}
    \end{frame}
    \begin{frame}[fragile, label = gauss_seidel.get_q_coeff]
        \IncMargin{1em}\begin{function}[H]
            \caption{getqCoeff(A, p)}
            \KwIn{Ma trận $A$, giá trị kiểm tra $p$}
            \KwOut{ $ q = \max\limits_{1 \leq i \leq n} \frac{\sum\limits_{j=i}^{n} \left\lvert A_{ij} \right\rvert}{1 - \sum\limits_{j=1}^{i-1} \left\lvert A_{ij} \right\rvert} $ nếu $p = 1$, $ \max\limits_{1 \leq i \leq n} \frac{\sum\limits_{j=1}^{i} \left\lvert A_{ji} \right\rvert}{1 - \sum\limits_{j=i+1}^{n} \left\lvert A_{ji} \right\rvert} $ nếu $p = -1$}
            \Begin{
                $q \longleftarrow 0$\;
                \For{$i=1$ \KwTo n}{
                    $ Q1 \longleftarrow 0, Q2 \longleftarrow 0 $\;
                    \lIf{$ p = 1 $}{$ Q1 \longleftarrow \sum\limits_{j=i}^{n} \left\lvert A_{ij} \right\rvert, Q2 \longleftarrow \sum\limits_{j=1}^{i-1} \left\lvert A_{ij} \right\rvert $}
                    \lElse{$ Q1 \longleftarrow \sum\limits_{j=1}^{i} \left\lvert A_{ji} \right\rvert, Q2 \longleftarrow \sum\limits_{j=i+1}^{n} \left\lvert A_{ji} \right\rvert $}
                    $ q \longleftarrow max(q, \frac{Q1}{1 - Q2}) $\;
                }
                \KwRet{$q$}
            }
        \end{function}\DecMargin{1em}

        \hyperlink{gauss_seidel.algo}{\textcolor{blue}{Trở về thuật toán chính}}
    \end{frame}
    \begin{frame}[fragile, label = gauss_seidel.iterate]
        \frametitle{Lặp - Đánh giá tiên nghiệm}
        
        \IncMargin{1em}\begin{function}[H]
            \caption{iterate($X_0$, B, T, S, q, p, $\omega, \varepsilon$)}
            \KwIn{Ma trận xấp xỉ đầu $X_{0}$, $B$, $T$, hệ số $S$, $q$, giá trị kiểm tra $p$, hệ số điều chỉnh $\omega$ và sai số $\varepsilon$}
            \KwOut{$X^{*}$ là ma trận nghịch đảo theo đánh giá tiên nghiệm}
            \Begin{
                $ qk \longleftarrow 1 $\; 
                $ X \longleftarrow X_{0} $\;
                $ X_{1} \longleftarrow \text{\hyperlink{gauss_seidel.next_iteration}{nextIteration}}(X_{0}, B, T, \omega) $\;
                $ predecessor\_norm \longleftarrow getNorm(X_{1} - X_{0}, p) $\;
                \BlankLine
                \While{$ \frac{qk * predecessor\_norm}{(1 - q)*(1 - S)} > \varepsilon $}{
                    $ X \longleftarrow \text{\hyperlink{gauss_seidel.next_iteration}{nextIteration}}(X, B, T, \omega) $ \;
                    $ qk \longleftarrow qk * q $ \;
                }
                \BlankLine
                \KwRet{X}
            }
        \end{function}\DecMargin{1em}
    
        \hyperlink{gauss_seidel.algo}{\textcolor{blue}{Trở về thuật toán chính}}
    \end{frame}
    \begin{frame}[fragile]
        \frametitle{Lặp - Đánh giá hậu nghiệm}

        \IncMargin{1em}\begin{function}[H]
            \caption{iterate($X_0$, B, T, S, q, p, $\omega, \varepsilon$)}
            \KwIn{Ma trận xấp xỉ đầu $X_{0}$, $B$, $T$, hệ số $S$, $q$, giá trị kiểm tra $p$, hệ số điều chỉnh $\omega$ và sai số $\varepsilon$}
            \KwOut{$X^{*}$ là ma trận nghịch đảo theo đánh giá hậu nghiệm}
            \Begin{
                $old\_X \longleftarrow X_{0} $\;
                $new\_X \longleftarrow \text{\hyperlink{gauss_seidel.next_iteration}{nextIteration}}(X_{0}, B, T, \omega) $\;

                \BlankLine
                \While{$ \frac{\lambda * q * \text{\hyperlink{jacobi.getNorm}{getNorm}}(new\_X - old\_X, p)}{1 - q} > \varepsilon $}{
                    $ old\_X \longleftarrow new\_X $\;
                    $ new\_X \longleftarrow \text{\hyperlink{gauss_seidel.next_iteration}{nextIteration}}(X, B, T, \omega) $\;
                }
                \BlankLine

                \KwRet{$new\_X$}
            }
        \end{function}\DecMargin{1em}

        \hyperlink{jacobi.algo}{\textcolor{blue}{Trở về thuật toán chính}}
    \end{frame}
    \begin{frame}[fragile, label = gauss_seidel.next_iteration]
        \IncMargin{1em}\begin{function}[H]
            \caption{nextIteration($old\_X, B, T, \omega$)}
            \KwIn{Ma trận $old\_X, B, T$, và hệ số $\omega$}
            \KwOut{Đưa ra ma trận tiếp theo thu được từ ma trận ban đầu $old\_X$ theo công thức lặp Gauss-Seidel}
            \Begin{
                $ new\_X \longleftarrow $ Ma trận không cấp $n$\;
                \For{$i=1$ \KwTo n}{
                    $ new\_X_{i} = \sum\limits_{j=1}^{i-1} B_{ij} * new\_X_{j} + \sum\limits_{j=i+1}^{n} B_{ij} * old\_X_{j} + T_{i} $\;
                }
                \KwRet{$(1 - \omega) * old\_X + \omega * new\_X$}
            }
        \end{function}\DecMargin{1em}

        \hyperlink{gauss_seidel.algo}{\textcolor{blue}{Trở về thuật toán chính}}

        \hyperlink{gauss_seidel.iterate}{\textcolor{red}{Trở về tiến trình lặp}}
    \end{frame}



    \begin{frame}[label = gauss_seidel.code]
        \frametitle{Chương trình}
        \begin{center}
            \includegraphics[scale = 0.4]{gausseidel-code-qr.png}

            \url{https://github.com/bu1th4nh/TALENTED-K64MI/blob/master/MI3040/report-code/gauss_seidel.py}
        \end{center}
        
    
    \end{frame}


    \section{Phân tích}
\begin{frame}
    \frametitle{Ưu, nhược điểm}

    \begin{center}
        \begin{tabular}{|c|c|c|c|}
            \hline
            Phương pháp &
              Newton &
              Lặp Jacobi &
              Lặp Gauss-Seidel \\ \hline
            Ưu điểm &
              \multicolumn{3}{c|}{\begin{tabular}[c]{@{}c@{}}Kiểm soát được sai số tính toán, sai số được cải thiện sau mỗi bước lặp\\ Tốc độ hội tụ nhanh trong một số trường hợp\end{tabular}} \\ \hline
            Nhược điểm &
              \begin{tabular}[c]{@{}c@{}}Khó tìm giá trị \\ xấp xỉ đầu $X_{0}$\end{tabular} &
              \multicolumn{2}{c|}{Yêu cầu ma trận phải chéo trội} \\ \hline
        \end{tabular}
    \end{center}

    $\longrightarrow$ Tìm giá trị xấp xỉ đầu cho $X_{0}$ trong phương pháp Newton?
    \pause

    \begin{itemize}
        \item Đặt $ X_{0} = \frac{A}{\left\lVert A \right\rVert_{1}\left\lVert A \right\rVert_{\infty}} $ \cite{PanReif}
        \item Sử dụng kết quả của các phương pháp tính trực tiếp ma trận nghịch đảo làm xấp xỉ đầu
    \end{itemize}

\end{frame}

\begin{frame}
    \begin{center}
        {\Huge \textbf{Câu hỏi?}}
    \end{center}
\end{frame}

\begin{frame}
    \frametitle{Tài liệu tham khảo}

    \nocite{*}
    \printbibliography    

\end{frame}

\begin{frame}
    \begin{center}
        \includegraphics[height = \textheight]{Ngan.png}
    \end{center}
\end{frame}



\end{document}