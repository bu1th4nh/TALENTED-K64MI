\documentclass[12pt,a4paper]{article}
\usepackage[utf8]{vietnam} % Sử dụng tiếng việt
\usepackage[top=2cm, bottom=2cm, left=3cm, right=2cm] {geometry} % Canh lề trang
\usepackage{graphicx} % Cho phép chèn hỉnh ảnh
\usepackage{fancybox} % Tạo khung box
\usepackage{indentfirst} % Thụt đầu dòng ở dòng đầu tiên trong đoạn
\usepackage{amsthm} % Cho phép thêm các môi trường định nghĩa
\usepackage{latexsym} % Các kí hiệu toán học
\usepackage{amsmath} % Hỗ trợ một số biểu thức toán học
\usepackage{amssymb} % Bổ sung thêm kí hiệu về toán học
\usepackage{amsbsy} % Hỗ trợ các kí hiệu in đậm
\usepackage{array} % Tạo bảng array
\usepackage{enumitem} % Cho phép thay đổi kí hiệu của list
\usepackage{subfiles} % Chèn các file nhỏ, giúp chia các chapter ra nhiều file hơn
\usepackage{titlesec} % Giúp chỉnh sửa các tiêu đề, đề mục như chương, phần,..
\usepackage{chngcntr} % Dùng để thiết lập lại cách đánh số caption,..
\usepackage{pdflscape} % Đưa các bảng có kích thước đặt theo chiều ngang giấy
\usepackage{afterpage}
\usepackage{capt-of} % Cho phép sử dụng caption lớn đối với landscape page
\usepackage{multirow} % Merge cells
\usepackage{fancyhdr} % Cho phép tùy biến header và footer
\usepackage{setspace}
\usepackage{parskip}
\usepackage[boxruled, lined]{algorithm2e} % Thêm mã giả

\usepackage{times}  % There's an impostor among us =)))


\onehalfspacing
\setlength{\parskip}{6pt}
\setlength{\parindent}{15pt}
\renewcommand{\baselinestretch}{1.5}
\renewcommand{\thesection}{\arabic{section}}

\begin{document}
    {\LARGE \textbf{Phụ lục C: Hướng dẫn sử dụng chương trình}}

    \section{Lưu ý trước khi sử dụng}

        \begin{itemize}
            \item Phần thuật toán chính được đóng gói trong các file thư viện, có mẫu là \texttt{lib\_*.py}. Các file này cung cấp các gói để chạy thuật toán thông qua các file chương trình chính \texttt{interface\_*.py} với * là tên phương pháp. Do đó, file thư viện phải để cùng thư mục với file chương trình chính.
            \item Các chương trình nhập đầu vào từ file \texttt{input.txt} và trả kết quả qua đầu ra chuẩn (standard output)
            \item Các chương trình và thuật toán có sử dụng thư viện \texttt{numpy} và \texttt{scipy} và càn cài đặt các thư viện này trước khi chạy bằng lệnh sau:
            \begin{verbatim}
                pip install numpy
                pip install scipy
            \end{verbatim}
        \end{itemize}

    \section{Hướng dẫn sử dụng chi tiết}

    \subsection{Phương pháp Newton}
    
    \par Các file trong thư mục: \texttt{Newton}
    \begin{itemize}
        \item \textbf{Chương trình chính:} \texttt{interface\_newton.py}
        \item \textbf{Gói thuật toán độc lập:} \texttt{lib\_newton.py}
        \item \textbf{File dữ liệu đầu vào:} \texttt{input.txt}
    \end{itemize}
    
    \par Các bước sử dụng
    \begin{itemize}
        \item \textbf{Bước 1:} Nhập ma trận vào file \texttt{input.txt}
        \item \textbf{Bước 2:} Chạy file \texttt{interface\_newton.py}. Chương trình sẽ tự động dò kích cỡ ma trận dựa theo ma trận người dùng nhập vào và báo lỗi nếu ma trận nhập vào không hợp lệ.
        \item \textbf{Bước 3:} Nhập sai số, sau đó chương trình sẽ đưa ra số lần lặp, ma trận nghịch đảo và kết quả nhân ngược. Trường hợp đầu vào không hợp lệ, chương trình sẽ báo lỗi và đưa ra ma trận NaN
    \end{itemize}

    \subsection{Phương pháp Jacobi}
    
    \par Các file trong thư mục: \texttt{Jacobi}
    \begin{itemize}
        \item \textbf{Chương trình chính:} \texttt{interface\_jacobi.py}
        \item \textbf{Gói thuật toán độc lập:} \texttt{lib\_jacobi.py}
        \item \textbf{File dữ liệu đầu vào:} \texttt{input.txt}
    \end{itemize}
    
    \par Các bước sử dụng
    \begin{itemize}
        \item \textbf{Bước 1:} Nhập ma trận vào file \texttt{input.txt}
        \item \textbf{Bước 2:} Chạy file \texttt{interface\_jacobi.py}. Chương trình sẽ tự động dò kích cỡ ma trận dựa theo ma trận người dùng nhập vào và báo lỗi nếu ma trận nhập vào không hợp lệ.
        \item \textbf{Bước 3:} Nhập sai số, sau đó chương trình sẽ đưa ra kiểu ma trận, số lần lặp, ma trận nghịch đảo và kết quả nhân ngược với 2 cách đánh giá tiên nghiệm và hậu nghiệm. Trường hợp đầu vào không hợp lệ, chương trình sẽ báo lỗi và đưa ra ma trận NaN
    \end{itemize}

    \subsection{Phương pháp Gauss-Seidel}
    
    \par Các file trong thư mục: \texttt{Gauss-Seidel}
    \begin{itemize}
        \item \textbf{Chương trình chính:} \texttt{interface\_gauss\_seidel.py}
        \item \textbf{Gói thuật toán độc lập:} \texttt{lib\_gauss\_seidel.py}
        \item \textbf{File dữ liệu đầu vào:} \texttt{input.txt}
    \end{itemize}
    
    \par Các bước sử dụng
    \begin{itemize}
        \item \textbf{Bước 1:} Nhập ma trận vào file \texttt{input.txt}
        \item \textbf{Bước 2:} Chạy file \texttt{interface\_gauss\_seidel.py}. Chương trình sẽ tự động dò kích cỡ ma trận dựa theo ma trận người dùng nhập vào và báo lỗi nếu ma trận nhập vào không hợp lệ.
        \item \textbf{Bước 3:} Nhập sai số, sau đó chương trình sẽ đưa ra kiểu ma trận, số lần lặp, ma trận nghịch đảo và kết quả nhân ngược với 2 cách đánh giá tiên nghiệm và hậu nghiệm. Trường hợp đầu vào không hợp lệ, chương trình sẽ báo lỗi và đưa ra ma trận NaN
    \end{itemize}


    
\end{document}