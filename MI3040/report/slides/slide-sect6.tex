\section{Phân tích}
\begin{frame}
    \frametitle{Ưu, nhược điểm}

    \begin{center}
        \begin{tabular}{|c|c|c|c|}
            \hline
            Phương pháp &
              Newton &
              Lặp Jacobi &
              Lặp Gauss-Seidel \\ \hline
            Ưu điểm &
              \multicolumn{3}{c|}{\begin{tabular}[c]{@{}c@{}}Kiểm soát được sai số tính toán, sai số được cải thiện sau mỗi bước lặp\\ Tốc độ hội tụ nhanh trong một số trường hợp\end{tabular}} \\ \hline
            Nhược điểm &
              \begin{tabular}[c]{@{}c@{}}Khó tìm giá trị \\ xấp xỉ đầu $X_{0}$\end{tabular} &
              \multicolumn{2}{c|}{Yêu cầu ma trận phải chéo trội} \\ \hline
        \end{tabular}
    \end{center}

    $\longrightarrow$ Tìm giá trị xấp xỉ đầu cho $X_{0}$ trong phương pháp Newton?
    \pause

    \begin{itemize}
        \item Đặt $ X_{0} = \frac{A}{\left\lVert A \right\rVert_{1}\left\lVert A \right\rVert_{\infty}} $ \cite{PanReif}
        \item Sử dụng kết quả của các phương pháp tính trực tiếp ma trận nghịch đảo làm xấp xỉ đầu
    \end{itemize}

\end{frame}

\begin{frame}
    \begin{center}
        {\Huge \textbf{Câu hỏi?}}
    \end{center}
\end{frame}

\begin{frame}
    \frametitle{Tài liệu tham khảo}

    \nocite{*}
    \printbibliography    

\end{frame}

\begin{frame}
    \begin{center}
        \includegraphics[height = \textheight]{Ngan.png}
    \end{center}
\end{frame}

